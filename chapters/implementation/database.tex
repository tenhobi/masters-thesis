\begin{figure}
    \centering
    \includegraphics[width=1\linewidth]{assets/implementation/erdiagram.png}
    \caption{Relational Schema}
    \label{fig:impementation:relationalschema}
\end{figure}

\begin{listing}
    \caption{Entity Framework Migration Tools}
    \label{listing:migration}
    \begin{minted}{shell}
# installation
dotnet tool install --global dotnet-ef

# create a migration
cd KingKarel
dotnet ef migrations add "Name of the migration"

# update the database
dotnet ef database update
    \end{minted}
\end{listing}

\section{Database}

According to the~design chapter~\ref{chapter:design}, PostgreSQL was used as a~database for the~game.
For local development, a~docker container with the~database is set up.
The~docker-compose file is located in the~server application project.

In the~chapter~\ref{design:conceptual} a~conceptual schema is described.
According to that schema, a~relational schema was created.
The~relational schema can be seen in the~figure~\ref{fig:impementation:relationalschema}.
The~strucute of relational schema is similar to the~structure of the~conceptual schema.
For unique identifiers integer data types are used.
For other attributes an~integer, a~string, or a~boolean data types are used.
Inside some of the~string attributes an~encoded JSONs are stored.

\subsection{Migrations}

Migrations manage the~database from the~server application.
Entity framework tools support the~creation and updates of migrations.
That is useful,\linebreak{}especially when the~database schema is created using the~ORM from the~code.
Each migration creates a~file with the~\mintinline{text}|Up()| and \mintinline{text}|Down()| methods.
These\linebreak{}methods allow migration tools to apply and undo a~migration.
Migration files are generated to \mintinline{text}|KingKarel/Migrations| directory.
Useful migration tool's commands can be seen in the~code~\ref{listing:migration}.
