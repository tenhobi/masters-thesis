\section{Game Mechanics}

The~game's principle is to fulfill the~individual game missions with the~training courses gradually.
Because the~game is designed as an~educational game, users must first register.

The~game contains separate courses called stories.
The~story contains\linebreak{}several missions.
These can be of different types: storytelling, learning, and game.
The~goal is to gradually go through the~missions, read the~story in the~storytelling missions, and get used to it.
Storytelling missions have step-by-step messages to structure the~story.
In learning missions, the~goal is to learn information and new concepts and terms or improve on them.
\linebreak
The~information in these missions should be more comprehensive and may overlap.
Game missions are used to test the~acquired knowledge, understand the~task, come up with solutions, and overcome the~challenge.

The~very concept of the~game is based on the~fact that the~player uses command blocks to move the~playable character of the~robot Karel, the~king of the~game's story.
The~player tries to perform various tasks with him.
Command blocks are a~visual programming interface that needs to be inserted into the~command list in the~correct order and, if necessary, nested correctly.
However, there is no single correct solution.
In addition, the~command blocks must be valid, i.e., all their mandatory attributes must be set.
There are several types of command blocks.

Command \mintinline{text}|move <direction>| moves Karel one square in the~chosen\linebreak{}direction.
However, Karel can only move through the~boxes designated for this purpose.
For example, it must not hit a~wall, outside the~grid, water, etc.
If Karel manages to hit such a~square, the~game ends with an~error.

With the~\mintinline{text}|put mark| command Karel places one mark on the~square he is standing on.
However, there cannot be an~unlimited number of marks in a~square.
If Karel tries to put a~mark on a~square where there is no more space, the~game ends with an~error.

With the~\mintinline{text}|grab mark| command Karel takes the~mark from the~square he is standing on.
However, such a~mark must be on the~square.
If Karel tries to take the~mark from the~square where the~mark was not, the~game ends with an~error.

With the~\mintinline{text}|if <condition>| command Karel asks a~question.
The~question may be whether it can move in that direction one square.
It may also be up to the~marks whether Karel can remove a~mark from the~current square or place it on it.
Additional commands are nested in this command.
These will only be performed if the~business is evaluated successfully.
If the~condition is not evaluated successfully, the~command and the~nested commands are skipped.

The~\mintinline{text}|while <condition>| command works similarly to the~command\linebreak{}\mintinline{text}|if <condition>|.
The~difference is that Karel repeats the process as long as the~condition is met.
If the~condition fails the~first time, the~statement is not executed, and no nested statements are executed.

Using the~appropriate commands correctly makes it possible to compose a~tree structure of command blocks that solve the~given game mission.
\mbox{After} pressing the~play button, the~game evaluation starts.
The~evaluation is\linebreak{}stepped through, and the~player has the~opportunity to see which current command is being executed.
The~player also has the~opportunity to see where Karel is and which other squares are updated, i.e., where the~number of marks has changed.
The~player can stop the~game by pressing the~stop button. 

While passing through the~game, Karel collects optional size and speed attributes.
The~size attribute indicates the~number of blocks that were used.
The~speed attribute indicates how many times any block has been used.

The~final dialog will be displayed if an~error is encountered during the~evaluation or if the~game is evaluated successfully.
This dialog takes two forms, either successful or unsuccessful.
The~successful dialog displays a~success \mbox{message} and the~status of the~optional size and speed attributes.
These \mbox{attributes} have a~specified limit.
If a~player scores better than these limits, they receive a~bonus point for each attribute.