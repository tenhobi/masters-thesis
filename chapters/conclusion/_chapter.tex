\begin{conclusion}

This thesis aimed to develop an~educational game for teaching programming, which focuses primarily on young people and uses gamification concepts.
A survey was conducted, followed by an~analysis of functional and non-functional requirements.
Based on the~analysis, research on existing similar games and applications that can be used to teach programming was made.
According to the~analysis and research, the~design of such a~game was done.
That was inspired by the~advantages of the~compared software, which tries to solve their shortcomings.
According to the~design, the~game was implemented and evaluated based on usability testing.

When designing the~implemented game \myAppName{}, the~architecture Clean Architecture was chosen, which inspires the~implementation parts mainly to the~easy extensibility and reusability of the~code.
The~Flutter framework was chosen to implement the~client part of the~game as a~cross-platform framework suitable for software development on mobile devices, desktops, and websites.
ASP.NET Web Api technology with the~use of the~C\# language was chosen to implement the~server part of the~game, which, thanks to its supporting technologies, enables the~creation of high-quality and robust implementation.
The~PostgreSQL relational database was selected as the~database.
Based on the~design, a~user interface was created that takes care of this education-first game's understandable and straightforward appearance.

\section{Acquired Experience}

% I gained a~lot of valuable experience from this project.
% I've explored both paths that work and paths that don't.
% In terms of experience with client software implementation, 
I successfully got acquainted with and implemented the~user interface with a~declarative approach, which the~Flutter framework encourages.
That was initially very non-intuitive compared to the~imperative approach.
Still, this approach has worked for me over time, and I understand that it has many advantages over the~imperative approach.

\pagebreak

To my surprise, I practically tried to perform usability testing, which went very well overall.
The~testers understood most of the~tasks and performed them very well.
The~game came to them visually, not bloated, and easy to understand.
The~testers even found a~few minor bugs or tips for improvement and one critical bug.

\section{Ideas on Future Development}

According to the~usability testing of the~prototype and the~fulfilled goals of the~work, the~developed could have a~potential.
That can be true, especially after further processing, better graphic and musical styling, and the~addition of more story and educational material.
The~game can also be used in primary or secondary schools to introduce programming and programming concepts.

Not-to-be-implemented features have already been analyzed in the~chapter~\ref{analysis:game:future-features}.
However, all the~features are exciting and would add other vital elements to the~game.
In addition, and from experience gained from the~development and response of testers, more improvements could expand the~game.
One idea is that block visual programming, as seen in the~game mission, may not be the~only concept of the~game.
Therefore, other types of missions could be added to the~game, where, for example, the~player would directly perform a~logical puzzle by clicking, moving, etc.
Thus, a~universal system for puzzles could be created that would allow them to be generated and checked using pre-prepared configurations.
Another type of mission could be with circuit and gate concepts, i.e., the~use of \mintinline{text}{and}, \mintinline{text}{or}, \mintinline{text}{not}, or \mintinline{text}{xor} gates to achieve the~desired output result.

One of the~first extensions could and should be to improve the~game's graphics and unique look-and-feel styling.
Although the~game consists of tutorials, the~passage through the~screens could be improved, for example, by using graphic elements and images that would color the~game and create a~pleasant atmosphere.
Related to this is the~passage of missions that are far too separate.
Ideally, the~game should be improved so that players can connect directly between missions, i.e., adding the~\textquote*{next} button to the~mission screens.

There are many ideas for expanding this game.
Some ideas focus on looks, others on game options.
The~game's future development is straightforward~--- gradually implement all appropriate improvements.

\end{conclusion}
