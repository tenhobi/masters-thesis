\section{Functional Requirements}

For the~context of this section, a~not-signed user is called \textquote*{a~user}, and a~signed user is called \textquote*{a~player}.
Moreover, a~player can use all features a~user can. 

Functional requirements describe individual requirements for the~functionality of the~game.
They describe actions or features users or players can use.
These requirements can be looked at as individual units, and they can form separate game modules.

Analysis of the~game that is the~base for requirements is described in more detail in section~\ref{analysis:game}.
Signing up, in, and out~-- as described in section~\ref{analysis:game:sign-up-in-out}~-- are crucial requirements that allow users to use the~locked-in features of the~game.
Looking up game information~-- as described in section~\ref{analysis:game:game-information}~-- is an~essential feature that allows all types of users and players to get the~available information like contacts, press data, guidelines, terms of use, etc.
Players also might need to take a~look at their profile or statistics~-- as described in sections~\ref{analysis:game:statistics} and~\ref{analysis:game:profile}.
The~game provides several courses called stories and their missions.
Missions can be of different types: storytelling, learning, or game missions.
Courses and missions are widely described in section~\ref{analysis:game:stories-and-missions}, and game mission and all its features, in particular, are described in section~\ref{analysis:game:game-missions}.
Some additional features are also described as future development ideas in section~\ref{analysis:game:future-features}.
However, these features are not included in the~list of functional requirements, as this list only contains elements for the~prototype that will be designed and implemented in the~game.

The following list shows the~analyzed functional requirements.
Requirements marked with \mintinline{text}{*} are beyond the scope of this thesis and will not be designed and implemented in more detail.

\pagebreak
\begin{enumerate}[label=\textbf{F\arabic*}, ref=\labelenumi]
    \myItem{Sign Up, In and Out} An~anonymous user must sign in to unlock courses and other in-game screens.
    If they do not have an~account, they can create one on the~sign-up screen.
    If they already have an~account, they can sign in to the~game on the~sign-in screen.
    If the~user is signed in, they can sign out of the~game, thus losing the~right to view the~in-game screens.

    \myItem{Game Information} An~anonymous user can display game information on the~about us screen, including approach, apps, contact, guidelines, help, info, press, privacy, and terms subscreens.

    \myItem{Statistics} A~player is shown the~result of the~game mission after its completion.
    They see whether they succeeded in the~mission and the~optional attributes \mintinline{text}|size| and \mintinline{text}|speed|.
    The~player can also view their summary statistics from all missions on the~statistics screen.
    On this screen, they can see the~name of the~mission, the~affiliation to the~story, whether it was successful, and the~optional attributes \mintinline{text}|size| and \mintinline{text}|speed|.

    \myItem{Profile} A~player can view their profile on the~profile screen.
    There they can see their nickname, name, email, and description.

    \myItem{Courses} A~player can view a~list of courses.
    This overview displays the~courses' names, descriptions, and how many missions they contain.
    On the~screen of individual courses, the~player then sees the~name of the~course and individual missions.
    The~player sees their name and description for each mission and can run them.
    In addition, they see a~state of the~game mission.
    The~state indicates whether the~game mission has been completed and, if so, whether the~optional size and speed attributes have been met.

    \myItem{Storytelling Missions} A~player can view the~storytelling mission.
    They can gradually click through the~story, introducing them to it.
    The~storytelling parts contain formatted text and pictures.

    \myItem{Learning Missions} A~player can view the~learning mission.
    They can read the~learning sections to learn the~concepts of the~game.
    The~learning sections contain formatted text and pictures. 
    
    \pagebreak
    \myItem{Game Missions} A~player can view and play the~game mission.
    It has several game mission features that will help players understand the~mission objectives and fulfill them.
    The~player can:
    \begin{enumerate}
        \item Display visual commands added to the~command list and can be moved in it using drag-and-drop.
        The~player can move new commands from the~palette.
        \item Display the~game grid, in which the~individual cells of the~game mission are displayed.
        There are cells for walkable cells and non-walkable cells.
        The~robot cannot walk outside the~marked grid.
        Walkable cells can display marks.
        \item Start a~game that processes visual commands and starts an~interactive robot walk through the~grid.
        The~player sees the~currently executed command and the~robot's current position in the~grid.
        \item At the~end of the~game, a~success or failure dialog will appear.
        Additionally, a~specific error message may be displayed if it fails.
        Optional \mintinline{text}|size| and \mintinline{text}|speed| attributes are displayed on success.
    \end{enumerate}

    \renewcommand{\labelenumi}{\textbf{F\arabic{enumi}}*}

    \myItem{Function Commands} A~player can use advanced commands such as functions.
    Functions can use other functions, and functions can also use themselves.
    The~player can create, edit and delete functions.

    \myItem{Machines} The~game has special cells~-- machines~-- which a~player can interact with.
    The robot can use commands and conditions which machines add to the~game.

    \myItem{Tools} The game has tools.
    These are entities that the~robot can pick up and use.
    They add commands and conditions to the~game.
    Tools can also be combined.

    \myItem{Practice Mission} A~player can view and complete a~practice mission that includes quizzes with yes-no or ABC questions.
    Questions must be answered correctly for the~player to continue.

    \myItem{Social Features} Players can communicate with each other through forums and chats.
    Players can view global statistics and public profiles.
    A~player can add other players to their friend list.

    \myItem{Creator Features} The~game has features for creating and managing missions and stories.
    A~creator can create and edit stories.
    They can also create and edit missions of all types.
    Various missions have adequate tools for their creation.

    \myItem{Manager Features} The game has features for editing texts on information screens.
    A~manager can edit the main screen, about-us screen, and about-us subscreens.

    \myItem{Teacher Features} The~game has an~additional type of user, a~teacher.
    Teachers can create classrooms where classes can be created.
    In a~given class, the~teacher can add and remove students for whom they can monitor performance and assign tasks.
    Students can view assigned tasks on the~classroom screen and compare themselves with other classmates if the~settings allow it.

    \myItem{Parent Features} The~game has an~additional type of user, a~parent.
    A~parent can monitor their children and give them assignments.

    \myItem{Random Challenge} The~game automatically creates a~random daily challenge.
    Players receive points for completing challenges, and players can compare with other players in the~statistics.
    The~game generates challenges automatically and correctly, including goals, grid, and description.
\end{enumerate}
