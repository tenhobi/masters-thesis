\section{Server Application}

The~design chapter~\ref{chapter:design} selected the~ASP.NET Web API in version 6.0.102 with C\# and the~Entity Framework in version 6.0.3 tool.

All services and repository interface implementations are registered in the~dependency injection tool provided by the~ASP.NET.
The~dependency injection tool allows automatic injection into constructors, thus abstracting the~process of creating most objects.
This is a~kind of magic compared to other languages and technologies, but it works well.
Registration is done in the~\mintinline{text}|Startup| class using \mintinline{text}|IServiceCollection| to register in the\linebreak{}\mintinline{text}|ConfigureServices| method.
Controllers are also registered in a~similar way using the~\mintinline{csharp}|services.AddControllers()| method does an automatic registration of all controllers.
Services and repositories are registered using \mintinline{csharp}|services.AddScopped<InterfaceType, ImplementationType>()|.

Setting up configurations for the~game is also made in the\linebreak{}\mintinline{text}|ConfigureServices|.
For configuration a~combination of standard public file \mintinline{text}|appsettings.json| and a~secret file \mintinline{text}|secrets.json| can be used.
The~structure of the~game's configuration can be seen in the~code~\ref{listing:serverconfig}.
The~\mintinline{text}|DatabaseUrl| and \mintinline{text}|JwtSecret| should be stored in a~secured file.

\begin{listing}
    \caption{Server Configuration}
    \label{listing:serverconfig}
    \begin{minted}{json}
{
    "AppSettings": {
        "DatabaseUrl": "<db url>",
        "JwtSecret": "<token>",
        "CorsOrigins": [
            "http://localhost:8080"
        ]
    }
}
    \end{minted}
\end{listing}

The~primary purpose of the~server application is to provide a~Web API that handles storing and fetching the~game's data.
It uses controllers to handle incoming HTTP requests.
Controllers use injected services that handle all logic.
Services usually use injected repositories that communicate with the~database using the~Language-Integrated Query (LINQ) to Entities.
LINQ to Entities can be used on the~\mintinline{text}|DbSet| properties of the~\mintinline{text}|DbContext| object.
An example of such usage in LINQ method syntax can be seen in the~code~\ref{listing:linq} that shows a~fetch of missions for a~specific story.

\begin{listing}
    \caption{LINQ to Entities}
    \label{listing:linq}
    \begin{minted}{csharp}
var missions = _dbContext.Stories
    .Include(s => s.Missions)
    .FirstOrDefault(s => s.Url == storyUrl)
    ?.Missions;
    \end{minted}
\end{listing}

\subsection{Saving the~Game Progress}

To save the~game progress the~HTTP PUT method on \linebreak\mintinline{text}|{storyUrl}/{missionUrl}| resource must be called.
It should contain \linebreak\mintinline{text}|GameProgressDto| data, which includes commands as an~encoded JSON in a~string, speed and size attributes as an~int, and the~completed attribute as a~bool.

Inside of the~controller, the~\mintinline{text}|StoryService| is called with the~\linebreak\mintinline{text}|SaveGameProgress| method.
Inside the~service, an~eponymous method is\linebreak{}called on the~\mintinline{text}|StoryRepository|.
Inside the~repository, if game progress\linebreak{}existed before, it is fetched.
If there is already an~existing record, its data get updated.
Otherwise, a~new \mintinline{text}|GameProgress| object must be created.
\linebreak
An important note is that the~created object must contain a~reference to the~user and game entities.
Therefore they must be fetched and used.

\subsection{Fetching Story's Missions}

The~\mintinline{text}|StoryRepository| repository also contains the~\mintinline{text}|GetMissions| method.
That is used to fetch missions of all types.
Because the~game mission can have an~associated \mintinline{text}|GameProgress| record for the~specified user, a~join has to be made.
LINQ contains two joins.
The~\mintinline{text}|Join()| method corresponds to the~SQL's inner join.
And the~\mintinline{text}|GroupJoin()| method that corresponds to the~SQL's left outer join.
For this purpose, the~\mintinline{text}|GroupJoin()| method was used.
The~code of the~left outer join can be seen in the~code~\ref{listing:groupjoin}.
Note that \mintinline{text}|missions| contains all missions of the~selected story.
It also uses the~\mintinline{text}|resultSelector| that merges the~joined data and converts the~entity from the~entity to do a~more suitable DTO.

\begin{listing}
    \caption{GroupJoin to Fetch Story's Missions}
    \label{listing:groupjoin}
    \begin{minted}{csharp}
var queryData = missions.OrderBy(m => m.Order)
    .GroupJoin(
        _dbContext.GameProgresses.Where(
            progress => progress.User.Id == userId
        ),
        m => m.Id,
        gp => gp.Game.Id,
        (mission, progresses) => GetMissionDtoFromGroupJoin(
            mission, progresses.SingleOrDefault()
        )
    ).ToList();
    \end{minted}
\end{listing}

DTOs cannot return different data based on inheritance.
For this purpose, the~\mintinline{text}|GetMissionDtoFromGroupJoin()| method returns the~\mintinline{text}|MissionsListDto|, which has two subtrees: either inside a~game property or a~learning property; the~other is null.
This way, inherited data can be returned in response to the~client from the~controller.
A better solution was not found.

\subsection{Tokens}

The~tokens were discussed in the~\ref{design:server:tokens} chapter.
\mintinline{text}|JwtService| service uses the~\mintinline{text}|Microsoft.AspNetCore.Authentication.JwtBearer| library in version 6.0.3 for token generation and validation.

For the~generation, it uses the~user's id as an~id claim, expiration is set to seven days, and it is signed using the~\mintinline{text}|HmacSha256Signature| algorithm.
\linebreak
Mentioned and other data create the~\mintinline{text}|SecurityTokenDescriptor| and the\linebreak{}\mintinline{text}|JwtSecurityTokenHandler| then creates the~token from it.

Validation of the~token uses \mintinline{text}|ValidateToken()| from the~\linebreak\mintinline{text}|JwtSecurityTokenHandler| object.

Clients must add JWT to the~authorization header like \linebreak\mintinline{text}|Authorization: Bearer <jwt-token>|.
A custom authentication scheme \linebreak\mintinline{text}|KingKarelAuthHandler| is used to validate the~token.
The~handler checks the~presence of the~authorization header and whether it starts with the~\textquote*{Bearer} string.
The~handler then uses the~\mintinline{text}|JwtService| that tries to validate the~token.
If the~token is valid, the~id claim is fetched from the~token and added as the~identity claim to the~\mintinline{text}|AuthenticationTicket|.
Controllers then can use these claims by accessing \mintinline{text}|User.Claims|.

\subsection{Passwords}

Passwords are hashed using the~\mintinline{text}|BCrypt.Net-Next| library in version 4.0.3.
The~library provides the~\mintinline{text}|BCrypt| object that has a~method \mintinline{text}|HashPassword()| that hashes a~password with a~generated salt.
It also provides the~\mintinline{text}|Verify()| method that requires a~password and a~hashed password arguments. 
\linebreak
Only hashed passwords are stored in the~database.
