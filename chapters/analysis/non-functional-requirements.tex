\section{Non-functional Requirements}

Non-functional requirements do not describe the~game's behavior or its in-game features.
They instead describe limitations and user expectations like the~ease of use.

\begin{enumerate}[label=\textbf{N\arabic*}, ref=N\arabic*]
    \myItem{Education} The~game is focused on teaching programming. It passes on the~necessary knowledge to its players, gradually develops their\linebreak{}awareness of programming concepts, and provides them with tasks in which they gain practical experience.
    
    \myItem{Comprehensibility} The~game is easy to understand and easy to use, even for inexperienced users.
    Inexperienced users should be able to learn basic concepts in some form of tutorial.
    For more complex concepts, there should be explanations in the~game.

    \myItem{Localizations} The~game supports English and is ready to be exten\-ded to other languages.
    The~game implementation supports a~localization system. 

    \myItem{Cross-platform} The~game is available on modern versions of web browsers and desktops on Windows and Linux platforms.
    The~game is developed with a~view to its easy expansion on mobile devices.
    At the~same time, all versions should have a~similar appearance and functionality, and data should be shared between different versions.

    \myItem{Architecture} The~game code is written clearly with the~appropriate architecture and conventions used, which will allow easy expansion of functionalities.    
\end{enumerate}
