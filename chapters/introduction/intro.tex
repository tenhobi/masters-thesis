In recent years people have used more and more games and applications, mainly due to the~arrival, further development, and massive popularization of~laptops, mobile phones, and other smart devices.

For clarity, this thesis uses the~term \emph{game} for a~type of a program whose purpose is to entertain or use the~user's knowledge to elevate the mechanics of the~game in comparison with the~term \emph{application} for a~type of a program whose primary purpose is helping a~user do things.
Therefore even though regular users might refer to some programs as applications, the~term game will be used instead in this thesis.

Although not many games design their concepts to be educational,\linebreak{}the~benefits of educational aspects are crystal clear.
It is always good to introduce as many fun concepts and mechanics as possible for educational applications.

Dedicated educational games introduce students to the~studied subject using classical approaches such as strictly focusing on the~topic with no extra features.
While using interactive methods fulfills its purpose, and students can learn the~concepts from these resources, they also face problems with students' attention spans because they feel bored with the~knowledge they are learning.

Instead of developing only dull and bland dedicated educational applications, many games use a~concept called \emph{gamification}.
Gamification is a~relatively new technique without a~clear definition.

According to the~article~\cite{dichev_2017_gamifying}, gamification in education is a~psychologically driven approach to increasing the~motivation and engagement of students by including game design principles.
The~article also states that research on gamification is diverse and that the~focus is mainly on empirical studies. However, the~article also \textcquote{dichev_2017_gamifying}{identified a~growing number of studies reporting empirical evidences for the~effectiveness of gamification in educational context.}
It~also states that the~understanding of gamification processes is limited, and it is \mbox{unknown} how to produce beneficial learning outcomes while avoiding harmful learning outcomes. 

\pagebreak
According to the~article~\cite{smiderle_2020_the}, gamification might affect participants\linebreak{}differently based on their personality traits.
The~study used students from\linebreak{}an~undergraduate programming class who were given gamified and\linebreak{}non-gamified learning environments.
The~article concludes that students with low agreeableness, low openness, and introverts showed remarkable improvement and that introverted students were more engaged than extroverted ones.

In the~article~\cite{nand_2019_engaging} the~study examines the~effects of gamification on numeracy at a~primary school level.
They selected three features from the~survey~--- challenges, feedback, graphics~--- and created two game versions presented to children.
One version with all features enabled and the~second one with\linebreak{}an~apparent lack of these features.
The study results showed that gamification \mbox{methods} were \textquote{more effective in enhancing children's learning and they found it more engaging.}

After reviewing mentioned articles, it is clear that gamification in education has been proven by empirical methods to provide a~significantly better and more engaging learning environment.
Mentioned studies show a~good direction in designing an~entertaining education game despite the~lack of \mbox{non-empirical} studies on psychological benefits.
