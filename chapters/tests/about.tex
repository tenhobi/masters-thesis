Game developers or software developers in general need to gain an unbiased view of how their software performs when used by a real user.
Finding out how the software is controlled, how easily, how efficiently and whether they would appreciate any changes is undeniably essential knowledge.
They will be the ones to use the software after all.

Usability testing (sometimes also referred to as \textquote*{user testing}), one of the user experience (UX) methods, is used to verify these questions.
According to \cite{moran_2019_usability}, its goals are to identify problems, find out preferences and find out how the target users behave.
An essential advantage can also be finding out which other features users would like to see and use.
For testing to be successful and for developers to find valuable data, unbiased people, ideally from the target group, must do testing.
That is because software developers have too much information about what to expect, where to expect it, and even how the software works.
They also typically have too much general technical knowledge and are therefore not suitable candidates for representing the average user; if the average user is the target group of the software.
Testing is also suitable if the development team has an experienced UI or UX designer.
Although they may have practical experience and knowledge, even these do not match the perceptions of the real user who sees the software for the first time.
\textcquote{moran_2019_usability}{The only way to get UX design right is to test it.}
In addition, usability testing can be used to point out or disprove a deficiency or need for improvement that most developers do not perceive as essential.

Usability testing has specific rules and procedures, as described in \cite{moran_2019_usability}.
It is undoubtedly impossible to turn on the software and let the user do anything.
It requires a moderator, typically one of the developers, or someone who has all the necessary information about the software.
Testers are also needed.
There don't have to be many testers; five participants are enough.
The moderator's task is to assign several tasks to the testers, which the testers gradually perform.
The moderator monitors the tester during this, and the tester gives feedback.
The moderator should be impartial and should not influence the testers in any way but may ask for specific details.

For testing to make sense, testers must perform real-world tasks.
According to \cite{moran_2019_usability}, they can be of various types, both specific and open.
Tasks should not use a specific software language but users' languages.
Using software-specific terms can affect testers.
The design of tasks should consider the choice of appropriate and comprehensible words that could unnecessarily confuse testers.
Tasks can be communicated to testers in spoken or written form.
It is a frequent requirement that the testers read the tasks aloud or reformulate them with their own words, allowing the moderator to verify that they have understood the task correctly.
When performing tasks, testers should also say their thoughts aloud so that the moderator knows what and how they are thinking.

Testing can be done in several forms.
As mentioned by \cite{moran_2019_usability}, the most common division is into personal vs. remote testing.
Remote testing is often moderated, but it does not have to be.
Moderated testing tries to approach personal testing, so the moderator and the tester are in the same video call, and the tester typically shares their screen, camera, etc.
Special tools and technologies such as eye tracking or mouse tracking can also be used.

Remote test methods with testers with a shared screen will be used to test the developed game \myAppName{}.
Tasks will be handed over in writing.
It will be tested in two phases.
The first phase will test 4 participants.
After completing the testing of the first phase, minor changes in the implementation of the game will apply, according to the participants' feedback.
A pre-test expectations are that participants will have difficulty finding a button to display a description of the game mission.
Then the testing of the second phase will be performed, also with 4 participants.
Finally, the testing of both phases will be evaluated, and feedback will be used to improve the implementation of the game in the future.
