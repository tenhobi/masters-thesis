\section{Use Cases}

A use case is a term that describes how actors use a program to achieve specific goals.
They describe and organize functional requirements from the end users' point of view.
They are sequences of events and interactions that users can easily follow to achieve specific goals.
Use cases have primary and can optionally have secondary scenarios of their flow.

Each use case has a name, triggering events, and the main flow of events.
The name contains a verb and a noun that express the goal of the use case.
The triggering events describe an initiation of the use case.
Each use case can have multiple triggering events.
And the main flow of events, step by step, describes the flow of interactions between the system and the actor.
Optionally, use cases can also have preconditions.
Preconditions are conditions that must be met before executing the use case.
Initiation and preconditions are a part of the flow, but they can be described separately.

Use cases of designed game recognize two actors: an Anonymous User, an actor who is not signed in to the game, and a Player, an actor who is signed in to the game.
The Player is an extension of the Anonymous User.
That means that the Player can use or execute everything the Anonymous User can use or execute.

\let\oldsubsection=\thesubsection
\renewcommand\thesubsection{UC\arabic{subsection}}

\pagebreak
\subsection{Sign Up}

This use case describes signing up for the game and is used by an Anonymous User who, if processed successfully, becomes a Player actor. If a Player actor tries to activate the use case, they are redirected to the main screen.

This use case starts when an actor navigates to the sign-up screen or activates the sign-up button.

\begin{enumerate}
    \item The game navigates the actor to the sign-up screen.
    \item The actor fills in their username, real name, email, password, and description into text inputs.
    \item The game determines the validity of the actor's data.
    \begin{enumerate}
        \item If no issues were found, the game signs in the actor and navigates them to their profile screen.
        The actor becomes a Player.
        \item If issues were found, the game announces the fail.
    \end{enumerate}
\end{enumerate}

\subsection{Sign In}

This use case describes the signing in the game, and only an Anonymous User can use it. If a Player actor tries to activate the use case, they are redirected to the main screen.
If processed successfully, the actor becomes a Player actor.

This use case starts when an actor navigates to the sign-in screen or activates the sign-in button.

\begin{enumerate}
    \item The game navigates the actor to the sign-in screen.
    \item The actor fills in their username and password into text inputs.
    \item The game determines the validity of provided username and the corresponding password.
    \begin{enumerate}
        \item If no issues were found, the game signs in the actor and navigates them to their profile screen. The actor becomes a Player.
        \item If issues were found, the game announces the fail.
    \end{enumerate}
\end{enumerate}

\subsection{Sign Out}

This use case describes signing out of the game, and only a Player actor can use it.
After processing the use case, the actor becomes an Anonymous User.

This use case starts when an actor activates the sign-out button.

\begin{enumerate}
    \item The game signs the actor out of the game.
    \item The actor is navigated to the main screen.
\end{enumerate}

\subsection{View Publicly Available Game Information}

This use case describes viewing the publicly available game information and other publicly available screens like the main screen and about-us screen and its subscreens.
It can be used by an Anonymous User actor, which also extends the use to the Player actor.

This use case starts when an actor navigates to the main screen, about-us screen, or its subscreens, activates the main-screen button, or navigates to or activates any other screen or buttons, leading to screens with publicly available information.

\subsubsection*{Scenario A -- Main Screen}

\begin{enumerate}
    \item The game navigates the actor to the main screen.
    \item There, the actor can view the information.
\end{enumerate}

\subsubsection*{Scenario B -- About Us Screen}

\begin{enumerate}
    \item The game navigates the actor to the about us screen (or one of its corresponding subscreens).
    \item There, the actor can view the information.
\end{enumerate}

\subsection{View Own Profile}

This use case describes viewing an actor's in-game profile.
A Player actor can only use it.
If an actor, not a Player, tries to activate the use case, they are redirected to the sign-in screen.

This use case starts when an actor navigates to the profile screen or activates the profile button.

\begin{enumerate}
    \item The game navigates them to their profile screen.
    \item There, the actor can view the profile.
\end{enumerate}

\subsection{View Own Statistics}

This use case describes viewing an actor's in-game statistics.
A Player actor can only use it.
If an actor, not a Player, tries to activate the use case, they are redirected to the sign-in screen.

This use case starts when an actor navigates to the statistics screen or activates the profile button.

\begin{enumerate}
    \item The game navigates the user to their statistics screen.
    \item There, the actor can view the statistics.
\end{enumerate}

\pagebreak
\subsection{View Courses}

This use case describes viewing a list of game courses~-- called stories~--, a specific story, and a list of the story's missions.
A Player actor can only use it.
If an actor, not a Player, tries to activate the use case, they are redirected to the sign-in screen.

This use case starts differently based on its scenarios.

\subsubsection*{Scenario A -- View Courses}

This scenario starts when an actor navigates to the stories screen or activates the stories button.

\begin{enumerate}
    \item The game navigates the actor to the stories screen.
    \item The actor can view a list of courses.
\end{enumerate}

\subsubsection*{Scenario B -- View Course}

This scenario starts when an actor navigates to the story screen or activates the specific story button.

\begin{enumerate}
    \item The game navigates the actor to the selected story screen.
    \item The actor can view a list of the story's missions and the story's name and description.
\end{enumerate}

\subsubsection*{Scenario C -- View Mission}

This scenario starts when, on the story screen, an actor activates the mission's item or button.

\begin{enumerate}
    \item The game displays a popup containing additional data about the mission.
    It displays the mission's name and description.
    \item If it is a storytelling mission, a \textquote{read} button is displayed.
    \item If it is a learning mission, a \textquote{learn} button is displayed.
    \item If it is a game mission, a \textquote{play} button is displayed.
    \item The actor can activate the displayed button to leave the screen.
\end{enumerate}

\pagebreak
\subsection{Use Missions}

This use case describes using and viewing the game mission.
A Player actor can only use it.
If an actor, not a Player, tries to activate the use case, they are redirected to the sign-in screen.

This use case starts when an actor navigates to the story's mission screen or activates the mission's item or button on the story screen.

\subsubsection*{Scenario A -- Storytelling Mission}

This scenario starts when an actor navigates to the storytelling mission or activates the storytelling mission's item or button on the story screen.

\begin{enumerate}
    \item The game navigates the actor to the storytelling mission screen.
    \item The actor is presented with a list of texts that can be shown step by step by clicking the next button.
    \item The back-to-story button is displayed after the actor progresses through all the texts.
    \item Using the back-to-story button, the actor can leave the screen to the mission's story screen.
\end{enumerate}

\subsubsection*{Scenario B -- Learning Mission}

This scenario starts when an actor navigates to the learning mission or activates the learning mission's item or button on the story screen.

\begin{enumerate}
    \item The game navigates the actor to the learning mission screen.
    \item The actor is presented with a learning text.
    \item A back-to-story button is displayed.
    \item Using the back-to-story button, the actor can leave the screen to the mission's story screen.
\end{enumerate}

\subsubsection*{Scenario C -- Game Mission}

This scenario starts when an actor navigates to the game mission or activates the game mission's item or button on the story screen.

\begin{enumerate}
    \item The game navigates the actor to the game mission screen.
    \item The actor is presented with a game grid, command list view, and buttons.
    \item The user can add or move commands, run or stop the current game or save the current game -- as described in custom use cases.
\end{enumerate}

\subsection{Save Current Game}

This use case describes the saving of the actor's current game.
A Player actor can only use it.

This use case starts when an actor activates the save button on the game mission screen.
Being on the game mission screen is a precondition of this use case.

\begin{enumerate}
    \item The game loads the current progress of the commands presented in the command list view.
    \item The game saves those data, together with completed, size, and speed attributes. 
\end{enumerate}

\subsection{Play Current Game}

This use case describes the playing of the actor's current game.
A Player actor can only use it.

This use case starts when an actor activates the play button on the game mission screen.
Being on the game mission screen is a precondition of this use case.

\begin{enumerate}
    \item The game locks the command list view, so the actor cannot interact with it. 
    \item The game loads the current progress of the commands presented in the command list view.
    \item The game process these commands.
    \item The game starts presenting a step-by-step progression of the game grid.
    The procession of each step is signalized by an arrow next to a command block.
    \item After a presentation is done, the game shows a dialog with optional size and speed challenge attributes.
    \begin{enumerate}
        \item If the game resulted in a success, a success dialog is presented with a corresponding status message.
        \item If the game resulted in a failure, a failure dialog is presented with a corresponding status message and related error message.
    \end{enumerate}
    \item The game unlocks the command list view.
    \item The game saves the current progression, as described in the separate use case. 
\end{enumerate}

\subsection{Stop Current Game}

This use case describes the stopping of the actor's current game.
A Player actor can only use it.

This use case starts when an actor activates the stop button on the game mission screen while the game is in progress.
Being on the game mission screen with a game in progress is a precondition of this use case.

\begin{enumerate}
    \item The game stops the progressing game.
    \item The game unlocks the command list view.
    \item The game does not save the progression and does not show the success or failure dialog. 
\end{enumerate}

\let\thesubsection=\oldsubsection

\subsection{Requirements Implementation Overview}

\begin{table}[]
    \centering
    \begin{tabular}{|c||c|c|c|c|c|c|c|c|}
        \hline
         & F1 & F2 & F3 & F4 & F5 & F6 & F7 & F8  \\\hline\hline
    UC1  & x  &    &    &    &    &    &    &     \\\hline
    UC2  & x  &    &    &    &    &    &    &     \\\hline
    UC3  & x  &    &    &    &    &    &    &     \\\hline
    UC4  &    & x  &    &    &    &    &    &     \\\hline
    UC5  &    &    &    & x  &    &    &    &     \\\hline
    UC6  &    &    & x  &    &    &    &    &     \\\hline
    UC7  &    &    &    &    & x  & x  & x  & x   \\\hline
    UC8  &    &    &    &    &    & x  & x  & x   \\\hline
    UC9  &    &    &    &    &    &    &    & x   \\\hline
    UC10 &    &    &    &    &    &    &    & x   \\\hline
    UC11 &    &    &    &    &    &    &    & x   \\\hline
    \end{tabular}
    \caption{Implementation of Use Cases and Compliance with Requirements}
    \label{table:usecases-requirements}
\end{table}

Use cases organize functional requirements. 
An overview of the implementation of use cases by functional requirements can be seen in table~\ref{table:usecases-requirements}.

Use cases distinguish two actors, an Anonymous User and a Player.
The Player actor is an extension of the Anonymous User actor.
The competence of both actors and relationships of individual use cases can be seen in the figure of the Use Case Diagram~\ref{fig:usecasediagram}.

\begin{figure}
    \centering
    \includegraphics[width=1\linewidth]{assets/design/usecasediagram.pdf}
    \caption{Use Case Diagram}
    \label{fig:usecasediagram}
\end{figure}
