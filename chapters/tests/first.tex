\section{First Phase}

The~moderator and testers met one at a~time for a~video call using Microsoft Teams.
Each tester was acquainted with the~process and purpose of testing.
They were explained and advised that they should share their idea aloud and that they should also read the~tasks aloud and, if necessary, reformulate them in their own words.
They were informed that the~moderator would monitor their actions and not actively communicate so as not to interfere with their decisions.
Each tester was sent a~task assignment in the~chat.

The~first testing phase was attended by representatives of children, parents, and teachers.
These testers were assigned a~five-digit code for anonymization and the~possibility of reference to specific testers in the~text.
The~testers \mintinline{text}|cbefa|, \mintinline{text}| 3977c|, \mintinline{text}|a6ef3|, and \mintinline{text}|fcb22| took part in the~first testing.

\subsection*{Task 1}

The~first task was to sign in to the~game, i.e., create an~account.

This task for all testers was understandable and straightforward.
The~button that moves them to the~registration screen is expected where it was.
Tester \mintinline{text}|cbefa| mentioned that they would appreciate the~additional password verification input.

\subsection*{Task 2}

The~second task was to go to the~game's main screen and then navigate to the~profile screen.

That was easy and understandable for most testers.
However, tester \mintinline{text}|3977c| did not understand what the~main screen was supposed to be and did not know how to get to it.
But in the~end, they successfully got to it.
Finding and going to the~profile screen was no problem for all testers.
Most testers also noticed that it is possible to use a~direct click on the~avatar icon as well as a~click on a~button in the~submenu.

\subsection*{Task 3}

The~third task was to find a~course screen and search for a~course that contained three missions.
They should then read the~name of this course.

This task was more complicated than the~previous ones.
A word that is not a~word from the~language of the~game was used in the~task.
It confused the~testers as they searched in vain for the~word.
The~biggest problem with this had testers \mintinline{text}|3977c| and \mintinline{text}|a6ef3|.
However, everyone eventually understood, for example, by the~exclusion method, that the~right thing they were looking for was \textquote*{stories}.
Tester \mintinline{text}|3977c| mentioned that they had a~problem with this task because they associated the~word \textquote*{stories} with a~function of the~same name from other applications, such as Instagram.

\subsection*{Task 4}

The~fourth task was to find out which missions the~story contained.
The~task was also to describe and explain how different mission graphics affect them and what they think they represent.

No one had a~problem with completing this task.
However, tester \mintinline{text}|3977c| did not understand the~graphics of a~learning mission and believed it meant success, not teaching.

\subsection*{Task 5}

The~fifth task was to open learning missions and read them.

This task came easy for all testers, and they all completed it correctly.
However, tester \mintinline{text}|fcb22| did not stop at learning missions and also tried to launch a~game mission.

\subsection*{Task 6}

The~sixth task concerns the~game itself.
The~goal is to open a~game mission and complete several subtasks.
They were to read the~description, explain how they understood it, and use a~visual programming tool to accomplish the~task.

The~first task was to find and read the~mission description.
Tester \mintinline{text}|3977c| said they did not know what the~task was and tried to deduce it from the~game grid.
Tester \mintinline{text}|cbefa| found the~button but mentioned that they would expect the~mission description to be displayed straight away.
However, all testers successfully found the~button and could read the~game's description.
They also managed to understand and explain it.

Another task was to meet the~objectives of the~mission using visual blocks.
Everyone succeeded, and everyone managed to understand the~meaning of the~blocks.
Tester \mintinline{text}|fcb22| tried to use \mintinline {text}|if| and \mintinline{text}|while| command blocks but did not understand their meaning, so they used a~different way to accomplish the~mission.

When the~game progress ended, the~results were displayed in the~dialog, along with optional \mintinline{text}|size| and \mintinline{text}|speed|.
Everyone except the~tester \mintinline {text}|fcb22| groped for their meaning.

Tester \mintinline{text}|fcb22| positively evaluated everything was easy to find, the~game mission was clear, and the~design was not bloated.

\subsection*{Task 7}

The~seventh task was to go to the~statistics screen and determine which missions were completed.

The~testers completed this task without any problems.
At first, tester \mintinline{text}|fcb22| misunderstood the~screen and confused the~story and the~mission.

\subsection*{Task 8}

The~eighth task was to empathize with a~journalist looking for information.
The~task was to get to the~page intended for them.

All testers got to the~about us screen.
This screen contains a~signpost to all other informational screens.
However, testers \mintinline{text}|3977c|, \mintinline{text}|a6ef3|, and \mintinline{text}|fcb22| did not know the~word \textquote*{press}, so they did not go to the~dedicated screen.

\subsection*{Task 9}

The~ninth task was to sign out of the~game and navigate to the~sign-in screen.

All testers completed this task without any problems.

\subsection*{Identified Shortcomings}

The~first phase of testing confirmed the~assumption that it would be appropriate to display a~description of the~game mission as soon as it is opened.
It also turned out that the~testers mostly did not understand the~meaning of the~\mintinline{text}|size| and \mintinline{text}|speed| attributes.
Apart from that, no other significant problems were found.
