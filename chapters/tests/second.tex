\section{Second Phase}

As in the first phase, the moderator and testers gradually made a video call using Microsoft Teams.
Each tester was acquainted with the process and purpose of testing.
They were explained and advised that they should share their idea aloud and that they should also read the tasks aloud and, if necessary, reformulate them in their own words.
They were informed that the moderator would monitor their actions and not actively communicate so as not to interfere with their decisions.
Each tester was sent a task assignment in the chat.

The second testing phase was attended by representatives of children, parents, and teachers.
These testers were assigned a five-digit code for anonymization and the possibility of reference to specific testers in the text.
The testers \mintinline{text}|2df9c|, \mintinline{text}|93750|, \mintinline{text}|a5961|, and \mintinline{text}|84152| took part in the second testing.

Unlike the first phase, the game mission's description was displayed immediately after launch.
A description of the \mintinline{text}|size| and \mintinline{text}|speed| attributes has also been added into the learning mission.

\subsection*{Task 1}

The first task was to sign in to the game, i.e., create an account.

This task was not a problem for any testers; everything was without problems.

\subsection*{Task 2}

The second task was to go to the game's main screen and then navigate to the profile screen.

This task was also very easy for all testers.
In addition, tester \mintinline{text}|2df9c| recommended that the application name serves as a button on the main page for more applications and games, but would like to highlight it more because it looks just like any other text.
Testers \mintinline{text}|84152| would like to have the name of the currently signed-in player in addition to the avatar icon in the menu.

\subsection*{Task 3}

The third task was to find a course screen and search for a course that contained three missions.
They should then read the name of this course.

This task was also successful for all testers.
However, tester \mintinline{text}|a5961|, at first, did not realize that \textquote*{stories} were the courses they were looking for.
Even so, everyone found the screen they were looking for in the end.

\subsection*{Task 4}

The fourth task was to find out which missions the story contained.
The task was also to describe and explain how different mission graphics affect them and what they think they represent.

The testers completed this task without any problems.
They all understood what missions are and their meaning according to the graphics used.

\subsection*{Task 5}

The fifth task was to open learning missions and read them.

This task was also easy for everyone.
Everyone went through the missions and understood the meaning of the story and the explanation of the commands of the game mission.

\subsection*{Task 6}

The sixth task concerns the game itself.
The goal is to open a game mission and complete several subtasks.
They were to read the description, explain how they understood it, and use a visual programming tool to accomplish the task.

As in the first testing phase, the task was more difficult for testers.
However, they all completed the mission.

Testers \mintinline{text}|2df9c|, \mintinline{text}|93750| and \mintinline{text}|a5961| had trouble adding blocks to the bottom of the command list.
The problem was with the addition itself when the area to be added was too small, and the testers dropped the blocks as if outside, albeit visually close.
Another problem was when a tester used the entire height of the panel, had to scroll down, and had to scroll again after adding.
So the problem was, or rather the expectation, that the panel would scroll automatically if the user dragged the block and pulled it down.
A similar problem with scrolling was when selecting the direction when part of the menu overlay overflows the bottom of the window, and the player had to scroll complicatedly. Even then, the selection was not entirely straightforward.

Tester \mintinline{text}|2df9c| also did not understand what marks are and what their meaning is.
Tester \mintinline{text}|84152| still had trouble understanding \mintinline{text}|size| and \mintinline{text}|speed| attribute, and suggested that the post-game dialog could also contain some help.

Tester \mintinline{text}|93750| also discovered an error while starting the game when the game tried to process an empty \mintinline{text}|while| command block, which caused an endless loop, which prevented the completion of queue processing.

\subsection*{Task 7}

The seventh task was to go to the statistics screen and determine which missions were completed.

Completing this mission was not a problem for the testers, and everyone completed the task.
The tester \mintinline{text}|2df9c|, like another tester in the first phase, first confused mission and story.

\subsection*{Task 8}

The eighth task was to empathize with a journalist looking for information.
The task was to get to the page intended for them.

All testers completed this task without any problems.
Testers \mintinline{text}|a5961| and \mintinline{text}|84152| mentioned that they would welcome a special button in the bottom menu.

\subsection*{Task 9}

The ninth task was to sign out of the game and navigate to the sign-in screen.

All testers managed to complete this task.
Tester \mintinline{text}|93750| was not sure of the sign-in icons that they did not understand, but they understood what to do.
In conclusion, tester \mintinline{text}|a5961| would welcome a video tutorial or overlay tutorial.

\subsection*{Identified Shortcomings}

The second phase of testing confirmed that displaying the description of the game mission immediately after launch is more appropriate.
A critical error was also found in processing blocks into the executable queue, where the program made an endless loop.
This bug has been fixed.
In addition, no other errors were found.
Players had less trouble understanding \mintinline{text}|size| and \mintinline{text}|speed| attributes, but it would probably be better to add some form of explanation to the dialog.
